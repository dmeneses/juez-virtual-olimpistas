Problems in Computer Science are often classified as belonging to a certain class of problems (e.g., NP, Unsolvable, Recursive). In this problem you will be analyzing a property of an algorithm whose classification is not known for all possible inputs.

Consider the following algorithm:

\texttt{1. input n\\
2. print n\\
3. if {\it n = 1} then STOP\\
4. \, if {\it n} is odd then {\it n $\leftarrow$ 3n + 1}\\
5. \, else {\it n $\leftarrow$ n/2}\\
6. GOTO 2\\
}

Given the input 22, the following sequence of numbers will be printed 22 11 34 17 52 26 13 40 20 10 5 16 8 4 2 1

It is conjectured that the algorithm above will terminate (when a 1 is printed) for any integral input value. Despite the simplicity of the algorithm, it is unknown whether this conjecture is true. It has been verified, however, for all integers n such that $0 < n < 1,000,000$ (and, in fact, for many more numbers than this.)

Given an input n, it is possible to determine the number of numbers printed (including the 1). For a given n this is called the cycle-length of n. In the example above, the cycle length of 22 is 16.

For any two numbers i and j you are to determine the maximum cycle length over all numbers between i and j.